\documentclass{article}
\usepackage[utf8]{inputenc}
\usepackage[english]{babel}
\usepackage{graphicx}
\usepackage{amsmath}
\graphicspath{{images/}}

\title{The User Behavior On Online Judge Website}
\author{yueyi zhuo}
%\institute{School of Mathematical,Sichuan Normal University}
\date{2017}

\begin{document}

\maketitle

\begin{abstract}

Abstract content.

\end{abstract}

\section{Introduction}

In 2016 Spring Festival, I solve some problems in Project Euler 
\footnote{In fact, I solve 70+ problems before my interest fade.}, 
which may be most famous online judge 
in Internet except Leetcode. I have noticed the interesting data structure related with rank and difficulty.
But I lacked enough powerful statistic tool to process and model it. 
In fact, I model it as a ordinary least squard which is not understood by me in the time. In the naive period,
linear regression model seems like a poor black box model that often be meeted in Machine Learning algorithm.

Now, partly because of study of course of Econometric, I totally understand the random model and the instance,
linear regression. 
Right, there're some random variable constrained by expectation function sharing same parameters, that's all.
A specific linear regression model correspond to many latent models specified indepent variable distribution,
but their expectation function are same. This lead to same result from estimator and model test.

My insight is given by some external user comment about their usage they use OJ. 
Someone claim that thay solve problem by order of the number of solved, but I solve problem by order of ID.
I'm fascinated by thinking that how to identify the percent of two type if we assume only the two type exists.
Since the particular problem represent a more general problem set, I would like to treat it as a bridge to that.

Let's formulate my ealiest idea, there're $N$ problems, $n$ solver. Let 
$s_{ij} \in \{ 0,1 \} \quad i \in \{ 1,\dots,n \} \; j \in \{ 1,\dots,N \}$ be whether solver $i$ solve the problem $j$.
So fixing solver $i$, we assume binary random vector $s_i = ( s_{i1} s_{i2} \dots s_{iN} )^T$ 
can be divided into some group which is determined by a specific distribution and they're independent.
Specially, I define the distribution $D_1$ which follow order of ID:


\begin{align*} % Notice align or align* is not used in math mode. They can be directly include in paragraph(normal) mode.
P(s_{i1} = 1) = p \\
P(s_{ij} = 1 | s_{i,j-1}=0) = 0 \\
P(s_{ij} = 1 | s_{i,j-1}=1) = p
\end{align*}


and define $D2$ which follow order of the number of solved:


\begin{align*}
P(s_{i,r(1)} = 1) &= p \\
P(s_{i,r(j)} = 1 | s_{i,r(j-1)}=0) &= 0 \\
P(s_{i,r(j)} = 1 | s_{i,r(j-1)}=1) &= p
\end{align*}


in which $r(i)$ is function that map the the rank of number of solved to id
(e.g. r(10)=20 means that the rank of number of solved of 20th problem is 10).
Obviously we can replace $r(i)$ to any onther order that mean some interesting thing to get other distribution.

Let $n_1,n_2$ be the individual number that follows $D_1,D_2$, hence $n_1+n_2=n$. Unfortunately, we only can 
observe the sum of every sample sequence:

$$
S_j = \sum_{i=1}^n s_{ij}
$$ 

Our data is only the a obseration of statistic $S_j$ exactly. It looks like:



\section{Motivation}

Motivation content.

\end{document}